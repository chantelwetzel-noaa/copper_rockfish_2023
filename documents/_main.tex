\DocumentMetadata{%
 %  uncompress, %only for debugging!!
  pdfversion=2.0,
  testphase={phase-II, tabular, graphic}%
 % testphase={phase-II,math, tabular, graphic}% TOC Does not work
   % testphase={phase-III,math}% TOC works
}
\tagpdfsetup{activate, tabsorder=structure}
% Use the following to fix bug in November 2023 download of LaTeX
\ExplSyntaxOn
\cs_generate_variant:Nn\__tag_prop_gput:Nnn{cnx}
\ExplSyntaxOff
\documentclass[11pt,
  english,
  letterpaper,
]{article}
\usepackage{sa4ss}
\usepackage{amsmath,amssymb,array}
\usepackage{booktabs}

% From tagged-template.latex
\usepackage{lmodern}
\usepackage{ifxetex,ifluatex}
\ifnum 0\ifxetex 1\fi\ifluatex 1\fi=0 % if pdftex
  \usepackage[T1]{fontenc}
  \usepackage[utf8]{inputenc}
  \usepackage{textcomp} % provide euro and other symbols
\else % if luatex or xetex
  \usepackage{unicode-math}
  \defaultfontfeatures{Scale=MatchLowercase}
  \defaultfontfeatures[\rmfamily]{Ligatures=TeX,Scale=1}
\fi

% Use upquote if available, for straight quotes in verbatim environments
\IfFileExists{upquote.sty}{\usepackage{upquote}}{}
\IfFileExists{microtype.sty}{% use microtype if available
  \usepackage[]{microtype}
  \UseMicrotypeSet[protrusion]{basicmath} % disable protrusion for tt fonts
}{}
\makeatletter
\@ifundefined{KOMAClassName}{% if non-KOMA class
  \IfFileExists{parskip.sty}{%
    \usepackage{parskip}
  }{% else
    \setlength{\parindent}{0pt}
    \setlength{\parskip}{6pt plus 2pt minus 1pt}}
}{% if KOMA class
  \KOMAoptions{parskip=half}}
\makeatother
\usepackage{xcolor}
\IfFileExists{xurl.sty}{\usepackage{xurl}}{} % add URL line breaks if available
\hypersetup{
  pdflang={en},
  hidelinks,
  pdfcreator={LaTeX via pandoc}}
\urlstyle{same} % disable monospaced font for URLs
\usepackage{longtable}
% Correct order of tables after \paragraph or \subparagraph
\usepackage{etoolbox}
\makeatletter
\patchcmd\longtable{\par}{\if@noskipsec\mbox{}\fi\par}{}{}
\makeatother
% Allow footnotes in longtable head/foot
\IfFileExists{footnotehyper.sty}{\usepackage{footnotehyper}}{\usepackage{footnote}}
\makesavenoteenv{longtable}
\usepackage{graphicx}
\makeatletter
\def\maxwidth{\ifdim\Gin@nat@width>\linewidth\linewidth\else\Gin@nat@width\fi}
\def\maxheight{\ifdim\Gin@nat@height>\textheight\textheight\else\Gin@nat@height\fi}
\makeatother
% Scale images if necessary, so that they will not overflow the page
% margins by default, and it is still possible to overwrite the defaults
% using explicit options in \includegraphics[width, height, ...]{}
\setkeys{Gin}{width=\maxwidth,height=\maxheight,keepaspectratio}
% Set default figure placement to htbp
\makeatletter
\def\fps@figure{htbp}
\makeatother
\setlength{\emergencystretch}{3em} % prevent overfull lines
\providecommand{\tightlist}{%
  \setlength{\itemsep}{0pt}\setlength{\parskip}{0pt}}
\setcounter{secnumdepth}{5}
\usepackage{lineno}
\usepackage[inline]{showlabels}
\ifxetex
  % Load polyglossia as late as possible: uses bidi with RTL langages (e.g. Hebrew, Arabic)
  \usepackage{polyglossia}
  \setmainlanguage[]{}
\else
  \usepackage[shorthands=off,main=english]{babel}
\fi

%Define cslreferences environment, required by pandoc 2.8
%https://github.com/rstudio/rmarkdown/issues/1649


\providecommand{\tightlist}{%
  \setlength{\itemsep}{0pt}\setlength{\parskip}{0pt}}

\usepackage{lineno}
\usepackage[inline]{showlabels}
\date{}
\newcommand{\trTitle}{}
\newcommand{\trYear}{2023}
\newcommand{\trMonth}{May}
\newcommand{\trAuthsLong}{truetruetrue}
\newcommand{\trAuthsBack}{Monk, M.H., C.R. Wetzel, J. Coates}
\newcommand{\trCitation}{
\begin{hangparas}{1em}{1}
\trAuthsBack{}. \trYear{}. \trTitle{}. \glsentrylong{pfmc}, Portland, Oregon. \pageref{LastPage}{}\,p.
\end{hangparas}}

\newcommand\includegraphicsifexists[2][width=\linewidth]{\IfFileExists{#2}{\includegraphics[#1]{#2}}{}}

\begin{document}

%%%%% Frontmatter %%%%%

% Footnote symbols in front matter
\renewcommand*{\thefootnote}{\fnsymbol{footnote}}

\small
\thispagestyle{empty}
\pagenumbering{roman}
\noindent
\begin{center}
\title{}
% \textnormal{\MakeTextUppercase{\trTitle{}}}
\vspace{1.5cm}
{\Large\textbf\newline{}}

\includegraphicsifexists[width=4in]{figure_title.png}
\vfill
by\\
Melissa H. Monk\textsuperscript{1}\\
Chantel R. Wetzel\textsuperscript{2}\\
Julia Coates\textsuperscript{3}\vfill
\textsuperscript{1}Southwest Fisheries Science Center, U.S. Department of Commerce, National Oceanic and Atmospheric Administration, National Marine Fisheries Service, 110 McAllister Way, Santa Cruz, California 95060\\
\textsuperscript{2}Northwest Fisheries Science Center, U.S. Department of Commerce, National Oceanic and Atmospheric Administration, National Marine Fisheries Service, 2725 Montlake Boulevard East, Seattle, Washington 98112\\
\textsuperscript{3}.na.character\vfill
\trMonth{} \trYear{}
\end{center}
\clearpage

% Fourth page: Colophon
\thispagestyle{empty}
\vspace*{\fill}
\begin{center}
\copyright{} \glsentrylong{pfmc}, \trYear{}\\
\end{center}
\par
\bigskip
\noindent
Correct citation for this publication:
\bigskip
\par
\trCitation{}
\clearpage

% Add TOC to pdf bookmarks (clickable pdf)
\pdfbookmark[1]{\contentsname}{toc}

% Table of contents page, lists of figures and tables
\tableofcontents\clearpage
\label{TRlastRoman}
\clearpage

% Table of contents
\newpage
\thispagestyle{empty} % to remove page number

% Settings for the main document
\pagenumbering{arabic}  % Regular page numbers
\pagestyle{plain}  % No page number on first page of main document, use 'empty'
\renewcommand*{\thefootnote}{\arabic{footnote}}  % Back to numeric footnotes
\setcounter{footnote}{0}  % And start at 1
\renewcommand{\headrulewidth}{0.5pt}
\renewcommand{\footrulewidth}{0.5pt}
%\pagestyle{fancy}\fancyhead[c]{Draft: Do not cite or circulate}

\newcommand{\lt}{\ensuremath <}
\newcommand{\gt}{\ensuremath >}

\linenumbers

\newcommand\CapeM{$40^\circ 10^\prime N$}
\newcommand\PtC{$34^\circ 27^\prime N$}
\newcommand\CAOR{$42^\circ 00^\prime N$}

\hypertarget{cpfv-samples}{%
\section{Appendix H. CPFV Fleet Description, Sampling, and Trip Types}\label{cpfv-samples}}

During the 2021 copper rockfish stock assessment meeting and during the pre-assessment workshop for the 2023 groundfish assessments, concern was raised regarding possible biases in the available data by trip types. This appendix describes teh sampling and available data from two recent years as an example of the relationship between the available data and catches for copper rockfish.

The CDFW conducts several surveys are part of the CRFS program, including an onboard observer survey, angler interviews, and a required CPFV logbook for each trip. The \href{https://nrm.dfg.ca.gov/FileHandler.ashx?DocumentID=62348\&inline}{CRFS sampling manual} outlines the sampling methodologies for each survey component. The state of California allows the filleting of fish at sea; to accurately identify rockfish to species and obtain measurements of whole fish the majority of angler interviews occur onboard the vessel for the CPFV fleet.

The \href{https://nrm.dfg.ca.gov/FileHandler.ashx?DocumentID=36136\&inline}{California Fisheries Recreational FIsheries Survey Methods} indicates the CPFV ``sampling goal is to sample onboard at the historical sampling frequency of two to five percent of estimated CPFV trips of interest (e.g., trips targeting groundfish, inshore and coastal pelagic species) at each CPFV landing and to sample other CPFV trip types dockside.'' The CPFV fleet is not required to speciate the catch in the logbooks. For this reasons the CPFV logbook data are rarely used to develop an index of abundance for groundfish species.

The CPFV fleet north and south of Point Conception are fundamentally different in terms of the vessels fishing, available target species, and also fishery regulations over the past twenty years. The CPFVs north of Point Conception targeting groundfish to not typically target other species on the same trip, with the exception being a half day bottomfish trip mixed with a half day of Dungeness crab pot fishing. The fleet north of Fort Bodega Bay, California is dominated by smaller 6-pack vessels with limited capacity. It may not be possible for a sampler to ride onboard a 6-pack vessel if it's at capacity. Overnight trips are extremely rare in northern California.

In southern California, the diveristy of targt species is higher an includes several bass species, state managed gamefish, coastal pelagics, and highly migratory species, as well as boats that fish in Mexican waters. The trip types in southern California include 1/2 trips where a CPFV will run two trips a day, 3/4 day trips to fishing grounds further offshore, such as the Channel Islands, and overnight trips that can access areas like San Nicholas in the southern Channel Island or may be combination trips fishing for tunas, but also catching rockfish limits. The diversity of trip types and mixed target trips in southern California pose a challenge for both sampling and managing the fisheries.

The tables below represent the sum of trips from 2018 and 2019 combined as they are the two years pre-COVID that best represent the CPFV fleet's activity. The total mortality in metric tons by District in 2018-2019 is similar to that over the recent time period (Table \ref{tab:catch-example}). Approximately 52\% of the total rockfish mortality is from northern California and 48\% from southern California. Within northern California, 50\% of the catch originates form District 3 and in southern Californis 81\% of the catch originates from fish landed in District 2.

We filtered the CPFV logbook data to trips that recorded at least one rockfish and the CPFV onboard observer data to trips that encountered at least 5\% rockfish species. We aggregated within each dataset across 2018 and 2019 and areas where necessary to ensure confidentiality requirements were met. The combined sampling rate from 2018-2019 was three percent of all CPFV trips in the data used. There were no observed trips in CRFS District 5 in 2018-2019 where 1,052 CPFV logbook records were submitted that recorded at least one rockfish.

There were 60 trips recorded as multi-day trips in northern California. This represented less then 1\% of the trips in northern California and we did not investigate the accuracy of these entries. In southern California, 14\% of CPFV logbook trips were recorded as multi-day trips and 1\% of the CRFS onboard observer trips were from multi-day trips.

In CRFS District 1, 79\% of the onboard observer trips were from 1/2 day trips, whereas in District 2 this represented 29\% of the observed trips. The vessels in District 2 typically offer fewer 1/2 day trips and fish the nearshore when weather precludes a trip to the northern Channel Islands.

When weather allows, the northern Channel Islands can be accessed from CRFS District 2 during a 3/4-1 day trip or a multi-day trip, depending on the port. For instance, the CPFVs from Oxnard, California access the northern Channel Island during multi-day trips, and the same areas are accessed by the 3/4-1 day boats out of Santa Barbara Landing.

Approximately 6\% of the logbooks from District 1 recorded a block number in Mexican waters as the primary location fished. Fewer than 10 trips that recorded block numbers along the mainland of southern California were recorded a multi-day trips. These were removed from the tables below.

One block number is recorded per CPFV logbook. In southern California, 70\% of the trips fishing the northern Channel Island and 57\% of the trips fishing the southern Channel Islands were recorded as single day trips.

\begingroup\fontsize{10}{12}\selectfont
\begingroup\fontsize{10}{12}\selectfont

\begin{longtable}[t]{c>{\centering\arraybackslash}p{2cm}>{\centering\arraybackslash}p{2cm}>{\centering\arraybackslash}p{2cm}}
\caption{\label{tab:catch-example}Catch in metric tons of copper rockfish by CRFS District for 2018 and 2019.}\\
\toprule
District & 2018 & 2019 & Total\\
\midrule
\endfirsthead
\caption[]{\label{tab:catch-example}Catch in metric tons of copper rockfish by CRFS District for 2018 and 2019. \textit{(continued)}}\\
\toprule
District & 2018 & 2019 & Total\\
\midrule
\endhead

\endfoot
\bottomrule
\endlastfoot
1 & 9.2 & 23.1 & 32.3\\
2 & 87.0 & 51.8 & 138.8\\
3 & 49.3 & 44.3 & 93.6\\
4 & 30.0 & 27.9 & 57.9\\
5 & 7.7 & 13.1 & 20.7\\
6 & 6.0 & 7.3 & 13.2\\*
\end{longtable}
\endgroup{}
\endgroup{}

\begingroup\fontsize{10}{12}\selectfont
\begingroup\fontsize{10}{12}\selectfont

\begin{longtable}[t]{c>{\centering\arraybackslash}p{2cm}>{\centering\arraybackslash}p{2cm}}
\caption{\label{tab:percent-observed}The percent of observed trips by trip type and area for 2018 and 2019.}\\
\toprule
Area & Single day & Multi day\\
\midrule
\endfirsthead
\caption[]{\label{tab:percent-observed}The percent of observed trips by trip type and area for 2018 and 2019. \textit{(continued)}}\\
\toprule
Area & Single day & Multi day\\
\midrule
\endhead

\endfoot
\bottomrule
\endlastfoot
Northern CA & 6\% & NA\\
Southern CA & 10\% & 1\%\\*
\end{longtable}
\endgroup{}
\endgroup{}

\begingroup\fontsize{10}{12}\selectfont
\begingroup\fontsize{10}{12}\selectfont

\begin{longtable}[t]{c>{\centering\arraybackslash}p{1.83cm}>{\centering\arraybackslash}p{1.83cm}>{\centering\arraybackslash}p{1.83cm}>{\centering\arraybackslash}p{1.83cm}>{\centering\arraybackslash}p{1.83cm}}
\caption{\label{tab:onboard-trips}Number of CPFV trips observed onboard during 2018-2019 by trip type and District.}\\
\toprule
District & 1/2 day trips & 3/4-1 day trips & Overnight trips & Percent observed with copper & Total copper observed\\
\midrule
\endfirsthead
\caption[]{\label{tab:onboard-trips}Number of CPFV trips observed onboard during 2018-2019 by trip type and District. \textit{(continued)}}\\
\toprule
District & 1/2 day trips & 3/4-1 day trips & Overnight trips & Percent observed with copper & Total copper observed\\
\midrule
\endhead

\endfoot
\bottomrule
\endlastfoot
1 & 391 & 104 & 4 & 24\% & 296\\
2 & 36 & 87 & 4 & 76\% & 783\\
3 & 86 & 55 & NA & 99\% & 864\\
4 and 6 & 10 & 69 & NA & 61\% & 150\\*
\end{longtable}
\endgroup{}
\endgroup{}

\begingroup\fontsize{10}{12}\selectfont
\begingroup\fontsize{10}{12}\selectfont

\begin{longtable}[t]{c>{\centering\arraybackslash}p{2cm}>{\centering\arraybackslash}p{2cm}}
\caption{\label{tab:logbook-triptype}Number of CPFV logbook entries with at least one rockfish grouped by region fishedand trip type during 2018-2019 by trip type and District.}\\
\toprule
Region & Multi-day & Single day\\
\midrule
\endfirsthead
\caption[]{\label{tab:logbook-triptype}Number of CPFV logbook entries with at least one rockfish grouped by region fishedand trip type during 2018-2019 by trip type and District. \textit{(continued)}}\\
\toprule
Region & Multi-day & Single day\\
\midrule
\endhead

\endfoot
\bottomrule
\endlastfoot
Mexico & 223 & 636\\
District 1 mainland & NA & 8324\\
Southern Channel Islands & 1170 & 1572\\
District 2 mainland & NA & 663\\
Northern Channel Islands & 1135 & 2600\\
Southern CA Offshore & 119 & 2243\\
District 3 & 58 & 5195\\
District 4 & NA & 3156\\
District 5 & NA & 1051\\
District 6 & NA & 1189\\*
\end{longtable}
\endgroup{}
\endgroup{}
\end{document}
