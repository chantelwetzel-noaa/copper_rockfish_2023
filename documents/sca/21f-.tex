% Options for packages loaded elsewhere
\PassOptionsToPackage{unicode}{hyperref}
\PassOptionsToPackage{hyphens}{url}
%
\documentclass[
]{article}
\usepackage{amsmath,amssymb}
\usepackage{lmodern}
\usepackage{iftex}
\ifPDFTeX
  \usepackage[T1]{fontenc}
  \usepackage[utf8]{inputenc}
  \usepackage{textcomp} % provide euro and other symbols
\else % if luatex or xetex
  \usepackage{unicode-math}
  \defaultfontfeatures{Scale=MatchLowercase}
  \defaultfontfeatures[\rmfamily]{Ligatures=TeX,Scale=1}
\fi
% Use upquote if available, for straight quotes in verbatim environments
\IfFileExists{upquote.sty}{\usepackage{upquote}}{}
\IfFileExists{microtype.sty}{% use microtype if available
  \usepackage[]{microtype}
  \UseMicrotypeSet[protrusion]{basicmath} % disable protrusion for tt fonts
}{}
\makeatletter
\@ifundefined{KOMAClassName}{% if non-KOMA class
  \IfFileExists{parskip.sty}{%
    \usepackage{parskip}
  }{% else
    \setlength{\parindent}{0pt}
    \setlength{\parskip}{6pt plus 2pt minus 1pt}}
}{% if KOMA class
  \KOMAoptions{parskip=half}}
\makeatother
\usepackage{xcolor}
\usepackage[margin=1in]{geometry}
\usepackage{graphicx}
\makeatletter
\def\maxwidth{\ifdim\Gin@nat@width>\linewidth\linewidth\else\Gin@nat@width\fi}
\def\maxheight{\ifdim\Gin@nat@height>\textheight\textheight\else\Gin@nat@height\fi}
\makeatother
% Scale images if necessary, so that they will not overflow the page
% margins by default, and it is still possible to overwrite the defaults
% using explicit options in \includegraphics[width, height, ...]{}
\setkeys{Gin}{width=\maxwidth,height=\maxheight,keepaspectratio}
% Set default figure placement to htbp
\makeatletter
\def\fps@figure{htbp}
\makeatother
\setlength{\emergencystretch}{3em} % prevent overfull lines
\providecommand{\tightlist}{%
  \setlength{\itemsep}{0pt}\setlength{\parskip}{0pt}}
\setcounter{secnumdepth}{-\maxdimen} % remove section numbering
\usepackage{booktabs}
\usepackage{longtable}
\usepackage{array}
\usepackage{multirow}
\usepackage{wrapfig}
\usepackage{float}
\usepackage{colortbl}
\usepackage{pdflscape}
\usepackage{tabu}
\usepackage{threeparttable}
\usepackage{threeparttablex}
\usepackage[normalem]{ulem}
\usepackage{makecell}
\usepackage{xcolor}
\ifLuaTeX
  \usepackage{selnolig}  % disable illegal ligatures
\fi
\IfFileExists{bookmark.sty}{\usepackage{bookmark}}{\usepackage{hyperref}}
\IfFileExists{xurl.sty}{\usepackage{xurl}}{} % add URL line breaks if available
\urlstyle{same} % disable monospaced font for URLs
\hypersetup{
  hidelinks,
  pdfcreator={LaTeX via pandoc}}

\author{}
\date{\vspace{-2.5em}}

\begin{document}

\hypertarget{fishery-dependent-data}{%
\subsection{Fishery-Dependent Data}\label{fishery-dependent-data}}

\hypertarget{commercial-fishery}{%
\subsubsection{Commercial Fishery}\label{commercial-fishery}}

\hypertarget{landings-and-discards}{%
\paragraph{Landings and Discards}\label{landings-and-discards}}

\hfill\break

Commercial landings prior to 1969 were extracted from the South West
Fishery Science Center (SWFSC) landings reconstruction database for
estimates from the California Catch Reconstruction
{[}@ralston\_documentation\_2010{]}. Landings in this database are
divided into trawl, non-trawl, and unknown gear categories. Regions 7
and 8 as defined by Ralston et al. {[}-@ralston\_documentation\_2010{]}
were assigned to south of Point Conception in California. Regions 2, 4,
and 5 are associated with areas north of Point Conception. Region 6 in
Ralston et al. {[}-@ralston\_documentation\_2010{]} included Santa
Barbara County (mainly south of Point Conception), plus some major ports
north of Point Conception. To allocate landings from Region 6 to the
areas north and south of Point Conception, we followed an approach used
by Dick et al. {[}-@dick\_status\_2007{]} for the assessment of cowcod.
Specifically, port-specific landings of total rockfish from the CDFW
Fish Bulletin series were used to determine the annual fraction of
landings in Region 6 that was north and south of Point Conception (Table
XX). Rockfish landings at that time were not reported at the species
level. Although the use of total rockfish landings to partition landings
in Region 6 is not ideal, we see this as the best available option in
the absence of port-specific species composition data. Landings from
unknown locations (Region 0) were allocated proportional to the landings
from known regions.

In September 2005, the California Cooperative Groundfish Survey (CCGS)
incorporated newly acquired commercial landings statistics from
1969-1980 into the CALCOM database. The data consisted of landing
receipts (``fish tickets''), including mixed species categories for
rockfish. In order to assign rockfish landings to individual species,
the earliest available species composition samples were applied to the
fish ticket data by port, gear, and quarter. These `ratio estimator'
landings are coded (internally) as market category 977 in the CALCOM
database, and are used in this and past assessments as the best
available landings for the time period 1969-1980 for all port complexes.
See Appendix A of Dick et al. {[}-@dick\_status\_2007{]} for further
details.

Commercial fishery landings from 1981-2022 were extracted from the
Pacific Fisheries Information Network (PacFIN) database (extracted
February 6, 2023). Landings were separated north and south of Point
Conception based on port of landing. Commercial landings for copper
rockfish were split into two fleets based on the fish landed condition,
live or dead, and aggregated across gear types (Table XX and Figure XX).
The selection of this fleet structure was based on potential differences
in selectivity by the fishery based on fish landed condition where the
live fish fishery may be targeting fish of particular sizes (i.e., plate
sized). The first year where fish were observed to be landed live for
copper rockfish in the area south of Point Conception was 1994.

Discarding was not estimated within the model. The commercial catches,
landings plus discards, were estimated external to the model based on
data from the West Coast Groundfish Observer Program (WCGOP) data
provided in the Groundfish Expanded Mortality Multiyear (GEMM) product.
The GEMM provides expanded estimates of landings, discard, and catches
based on observed trips by sector split north and south of 40\(^\circ\)
10' N. lat. for the commercial fishery. Estimated landings and discards
south of 40\(^\circ\) 10' N. lat. from select sectors (LE Fixed Gear DTL
- Hook and Line, Nearshore, CS - Hook and Line, OA Fixed Gear - Hook and
Line, OA Fixed Gear - Pot, and LE Fixed Gear DTL - Pot) were used to
calculate a discard rate (total discard divided by the sum of landings
and discards by year) for 2002-2021. The annual discard rates were
applied to the total landings by year to calculate catches for both
areas south and north of Point Conception. The median discard rate south
of 40\(^\circ\) 10' N. lat. from the select sectors between 2002-2021 in
the GEMM was 3 percent. This discard rate was applied to landings
between 1916-2001 and 2022 to determine catch by year. The assumptions
around the discard rate by year had limited impact to the assumed total
catches given the limited scale of removals by the commercial fishery
for copper rockfish. Across all years, 1916-2022, the landings were
increased by 2-3 percent by area (11 mt south of Point Conception and 26
mt north of Point Conception) to calculate the total catches.

\hypertarget{recreational-fishery}{%
\subsubsection{Recreational Fishery}\label{recreational-fishery}}

\hypertarget{landings-and-discards-1}{%
\paragraph{Landings and Discards}\label{landings-and-discards-1}}

\hfill\break

The recreational fishery is the main source of exploitation of copper
rockfish across California. The recreational catches of copper rockfish
south of Point Conception in California waters peaked in the late 1970s
and early 1980s. Catches declined in the 1990s and early 2000s. The
removals remained relatively low until 2015. Catches begun to increase
in 2015, likely due to changes in harvest specifications
{[}@cope\_data-moderate\_2013{]}. The catches decreased in 2020 due to
COVID-19 impacts and remained relatively low in 2021 and 2022 due to
reductions in the sub-bag limits in California for copper rockfish. The
recreational fishery was split into two fleets based on fishing type
(termed `modes'), a commercial passenger fishing vessel (CPFV,
party/charter mode) fleet and a combined private or rental boats (PR
mode) and shoreside (man-made and beach/bank modes) fleet. The catches
associated with the shoreside mode for copper rockfish are limited and
did not justify a separate fishing fleet within the model.

Recreational landing estimates from 1928 to 1980 were obtained from the
historical reconstruction {[}@ralston\_documentation\_2010{]}. The
historical landings reconstruction split removals north and south of
Point Conception and by recreational modes. CPFV landings of all
rockfish were based on logbook data (which do not report rockfish to the
species level), scaled by compliance estimates, while total recreational
landings from PR vessels were based on a combination of the relative
catch rates observed in the CPFV fleet and a linear ramp between catch
estimates in the early 1960s and those in the early 1980s (as described
in Ralston et al. {[}-@ralston\_documentation\_2010{]}). The species
composition of rockfish landings was estimated using a combination of
the 1980s Marine Recreational Fisheries Statistics Survey (MRFSS) data
as well as limited CPFV mode species composition data from onboard
observer programs in the late 1970s (south of Point Conception) and
dockside recreational creel surveys in the late 1950s and early 1960s
(north of Point Conception).

Recreational removals from 1981-1989 and 1993-2003 were obtained from
MRFSS downloaded from the Recreational Fisheries Information Network
(RecFIN). Historically, copper rockfish were occasionally referred to as
whitebelly rockfish in select California areas. MRFSS catches were
pulled for both species names. MRFSS includes estimates of removals for
1980. However, due to inconsistencies in the estimates of this year in
MRFSS, likely due to it being the first year of the survey with low
sample sizes, the value for recreational landings from the historical
reconstruction were used {[}-@ralston\_documentation\_2010{]}.

Some known issues with the MRFSS estimates include 1) a change in the
spatial definition of California subregions after 1989, 2) missing or
imprecise estimates of catch in weight for some strata that reported
catch in numbers, and 3) a hiatus in sampling from 1990-1992 (all modes)
and also 1993-1995 in the party/charter mode north of Point Conception.
The STAT attempted to address each of these issues, as described below.
CRFS estimates from 2004 were also included in the MRFSS analysis, as
they were not available on the current RecFIN website but are included
with the MRFSS catch estimate tables

The MRFSS definition of ``Southern California'' included San Luis Obispo
County between 1981-1989, requiring the catches from this county to be
split out and removed from the recreational catch south of Point
Conception. The MRFSS catches between southern and northern California
were adjusted in a similar fashion as previous assessments split at
Point Conception. Albin et al. {[}-@albin\_effort\_1993{]} used MRFSS
data to estimate catch at a finer spatial scale from the
California/Oregon border to the southern edge of San Luis Obispo (SLO)
County. Over the period 1981-1986, numbers of copper rockfish landed in
SLO County were found to be approximately one third (0.317) of the
numbers of copper rockfish landed in all California counties north of
SLO County {[}@albin\_effort\_1993{]}. Therefore, to approximate catches
north and south of Point Conception from 1980-1989, the STAT reduced the
`southern' subregion annual catch (which included SLO County) from
1980-1989 by 0.317 during the same period, and added this amount to the
northern subregion catch. On average, this `moves' the estimated SLO
County catch from the southern region to the northern region from
1980-1989, creating a spatially consistent time series of landings over
the entire time series.

The STAT chose to use catch in terms of weight (WGT\_AB1 column) within
MRFSS. The catch weights were converted from kilograms to metric tons
and any records with missing catch weights were examined. The number of
records with missing catch weights for copper rockfish in MRFSS were
limited (only 18 out of 713). The missing catch weights were imputed
based on the number of fish (TOT\_CAT column) and the calculated average
fish weight by year and area north and south of Point Conception.

MRFSS sampling was halted from 1990-1992 due to funding issues. The
survey resumed in 1993 in all modes, except for the PC boat mode which
resumed in 1996 for counties north of Santa Barbara County. To produce
catch estimates for the missing subregion, mode, and year combinations
linear interpolations were used to fill in the missing data.

Two additional revisions were applied to select years and modes in the
MRFSS data based on conversations with California Department of Fish and
Wildlife (CDFW). The catches for the PR mode north of Point Conception
in MRFSS for 1981 were 50 to 90 percent greater than the catches in 1980
and 1982, respectively. The high catches in this year were assumed to be
a result of issues in the catch expansions due to limited sampling. The
catches for the PR fleet were revised downward to be equal to the
average removals in surrounding years (1979, 1980, 1982, and 1983). The
catches in MRFSS south of Point Conception in 1987 were identified as
abnormally low by CDFW (John Budrick, pers. communication, 13 to 27
percent of catches in 1986 and 1988) which was due to no catch
information for waves 1-3 (January - June) for either mode. Absence of
data in 1987 for these waves was not observed across other rockfish
species in southern California indicating that the absence of catch data
was likely not due to closures in the fishery. The catches for this year
and mode were set equal to the average catch by mode 2 years before and
after 1987.

Recreational landings from 2004-2019 were obtained from California
Recreational Fisheries Survey (CRFS) available on RecFIN. This survey
improves upon the MRFSS sampling design, employing higher sampling rates
and producing estimates with finer spatial and temporal resolution. CRFS
also employs onboard CPFV observers, providing spatially referenced,
drift-level estimates of catch and discard for a subset of anglers on
observed groundfish trips. Any CRFS records of fish caught in Mexican
waters were removed and catch estimates were split north and south of
Point Conception for each fleet. Due to database issues, catches for
2004 are currently not available on RecFIN. The catches for this year
were set equal to data pulled in 2021 for the previous assessment of
copper rockfish.

The recreational catches for 2020-2022 were provided directly by CDFW.
During 2020 due to the COVID-19 pandemic dockside sampling by observers
was halted April through June leading to missing catch data within the
CRFS database for this period. Additionally, when sampling resumed a
large number of rockfish catches were not identified to species
(recorded as rockfish genus only) for the remainder of 2020 and 2021 due
to social distancing for health and safety. CDFW provided proxy catch
values that allocated a subset of the rockfish genus removals by
recreational mode north and south of Point Conception for these years.
Finally, the completed catch estimates for 2022 were not available for
CRFS on RecFIN by the data deadline for this assessment and estimates
were provided directly to the STAT from CDFW.

MRFSS and CRFS both provide estimates of total mortality which combine
observed landings plus estimates of discarded fish using depth-dependent
mortality rates. While the recreational removals from the historical
reconstruction from 1928-1980 account for only landed fish. There is
limited information on historical discarding in the recreational
fishery. A report by Miller and Gotshall {[}-@miller\_ocean\_1965{]}
looked at the number of retained and discard fish in the recreational
fishery in California for a select year which showed essentially no
discarding of copper rockfish. Based on that no additional discards were
applied to the historical data between 1926-190.

\end{document}
